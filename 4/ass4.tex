\documentclass[12pt]{article}
\usepackage{amsfonts,amsmath,amssymb,graphicx,url}
\usepackage{fullpage}


% Old Stuff
%%\oddsidemargin=0.15in
%%\evensidemargin=0.15in
%%\topmargin=-.5in
%%\textheight=9in
%%\textwidth=6.25in

\setlength{\oddsidemargin}{.25in}
\setlength{\evensidemargin}{.25in}
\setlength{\textwidth}{6.25in}
\setlength{\topmargin}{-0.4in}
\setlength{\textheight}{8.5in}

\newcommand{\heading}[5]{
   \renewcommand{\thepage}{\arabic{page}}
   \noindent
   \begin{center}
   \framebox{
      \vbox{
    \hbox to 6.2in { {\bf CSE/AMS 547 Discrete Mathematics}
     	 \hfill #2 }
       \vspace{4mm}
       \hbox to 6.2in { {\Large \hfill #5  \hfill} }
       \vspace{2mm}
       \hbox to 6.2in { {\it #3 \hfill #4} }
      }
   }
   \end{center}
   \vspace*{4mm}
}

\newcommand{\handout}[3]{\heading{#1}{#2}{Instructor:
David Gu}{}{#3}}

\setlength{\parindent}{0in}
\setlength{\parskip}{0.1in}

\begin{document}
\handout{2}{December 4, 2017}{Homework 4}

\textbf{Sharad Sridhar - 111492675}

\textbf{Due by Monday, Dec. 4, 11:59pm.}

\noindent
{\bf Problem 5.18}

\textit {Solution :}

- 5.35 can be written as : 

$\binom{r}{k}\binom{r-\frac{1}{2}}{k}=\frac{\binom{2r}{2k}\binom{2k}{k}}{2^{2k}}$

Also, the expansion of $\binom{r}{k}$ can be written as:

$\binom{r}{k} = \frac{r!}{(r-k)!(k!)} = \frac{(r)(r-1)(r-2)...(r-k+1)}{k!}$

Applying this expansion to the given equation:

$\binom{r}{k}\binom{r-1/3}{k}\binom{r-2/3}{k}$

$= (\frac{(r)(r-1)(r-2)...(r-k+1)}{k!})(\frac{(r-1/3)(r-1/3-1)...(r-1/3-k+1)}{k!})(\frac{(r-2/3)(r-2/3-1)(r-2/3-2)...(r-2/3-k+1)}{k!})$

$= (\frac{(r)(r-1)(r-2)...(r-k+1)}{k!})(\frac{(r-1/3)(r-4/3)...(r-k+2/3)}{k!})(\frac{(r-2/3)(r-5/3)(r-8/3)...(r-k+1/3)}{k!})$

$= (\frac{(r)(r-1/3)(r-2/3)...(r-k+1)(r-k+2/3)(r-k+1/3)}{k!k!k!})$

We can multiply this throughout by $3^{3k}$

$= {\frac{(3r)(3r-1)(3r-2)(3r-3)...(3r-3k+3)(3r-3k+2)(3r-3k+1)}{k!k!k!}}{\frac{1}{k!k!k!3^{3k}}}$

$= (3r)(3r-1)(3r-2)(3r-3)...(3r-3k+3)(3r-3k+2)(3r-3k+1)(\frac{1}{k!k!k!3^{3k}})$

$= (\frac{(3r)!}{3r-3k})(\frac{1}{k!k!k!3^{3k}})$

Multiplying and dividing throughout by $(3k)!$ and $(2k)!$, we get

$= (\frac{(3r)!}{3r-3k})(\frac{1}{k!k!k!3^{3k}})(\frac{(3k)!}{(3k)!})(\frac{(2k)!}{(2k)!})$

$= (\frac{(3r)!}{(3r-3k)!(3k)!})(\frac{(3k)!}{(2k!)k!})(\frac{(2k)!}{k!k!})(\frac{1}{3^{3k}})$

$= \frac{(\binom{3r}{3k})(\binom{3k}{2k})(\binom{2k}{k})}{3^{3k}}$

which is the required result

\medskip

\medskip

\noindent
{\bf Problem 5.19}

\textit {Solution :}

5.58 is : $B_t(z) = \sum_{k\geq 0}^{} (tk)^{\underline{k-1}}(\frac{z^k}{k!})$

Expanding this, we get:

$= \sum_{k\geq 0}^{}{\frac{(tk)(tk-1)(tk-2)...(tk-k+2)(z^k)}{k!}}$

Multiplying and dividing throughout by $(tk-k+1)$, we get:

$= \sum_{k\geq 0}^{}{\frac{(tk)(tk-1)(tk-2)...(tk-k+2)(tk-k+1)(z^k)}{(tk-k+1)k!}}$

$= \sum_{k\geq 0} \frac{(\binom{tk}{k})(z^k)}{(tk-k+1)}$

We have equation 5.60 as:

$B_t(z)^r = \sum_{k\geq 0} {\binom{tk+r}{k}}{\frac{r}{tk+r}}{z^k}$

Let r=-1 and substitute t with 1-t and z with -z :

$B_{1-t}(-z)^-1 = \sum_{k\geq 0} {\binom{k-tk-1}{k}}{\frac{-1}{k-tk-1}}{-z^k}$

Applying upper negation to the form above, we have:

$= \sum_{k\geq 0} {\binom{k-tk-1}{k}}{\frac{-1}{k-tk-1}}{-1^k}{z^k}$

$= \sum_{k\geq 0} {\binom{k-tk-1}{k}}{\frac{-1}{k-tk-1}}{(-1^k)}{z^k}$

$= \sum_{k\geq 0} {\binom{k-(k-tk-1)-1}{k}}{\frac{-1}{k-tk-1}}{z^k}$

$= \sum_{k\geq 0} {\binom{tk}{k}}{\frac{1}{tk-k+1}}{z^k}$

$= \sum_{k\geq 0} \frac{(\binom{tk}{k})(z^k)}{(tk-k+1)}$

Which is equal to the previous result. Hence, proved...

\medskip

\medskip

\noindent
{\bf Problem 5.40}

\textit {Solution :}

$\sum_{j=1}^{m} {-1^{k+1}}{\binom{r}{j}}{\sum_{k=1}^{n}\binom{-j+rk+s}{m-j}}$

Swapping the summations of j and k, we have:

$= \sum_{j=1}^{m} {-1^{j+1}}{\binom{r}{j}}{\sum_{k=1}^{n}(-1)^{m-j}\binom{(m-j)-j+rk+s-1}{m-j}}$

$= \sum_{j=1}^{m} {-1^{j+1}}{\binom{r}{j}}{\sum_{k=1}^{n}(-1)^{m-j}\binom{m-rk-s-1}{m-j}}$

Combining the two summations:

$= \sum_{k=1}^{n}\sum_{j=1}^{m} {(-1)^{j+1}}(-1)^{m-j}{\binom{r}{j}}{\binom{m-rk-s-1}{m-j}}$

$= \sum_{k=1}^{n}\sum_{j=1}^{m} {(-1)^{m+1}}{\binom{r}{j}}{\binom{m-rk-s-1}{m-j}}$

$= \sum_{k=1}^{n}\sum_{j=1}^{m}{{\binom{r}{j}}{\binom{m-r(k-1)-s-1}{m}}}-{\binom{m-r(k)-s-1}{m}}$


$= (-1)^{m+1}\sum_{k=1}^{m}{{\binom{r}{j}}{\binom{m-r(k-1)-s-1}{m}}}-{\binom{m-r(k)-s-1}{m}}$

$= (-1)^{m}\sum_{k=0}^{m}{{\binom{m-r(k)-s-1}{m}}}-{\binom{m-s-1}{m}}$

$= (-1)^{m}{{\binom{m-rn-s-1}{m}}}-{\binom{m-s-1}{m}}$

$= {\binom{rn+s}{m}}-{\binom{s}{m}}$

Which is the required result

\medskip

\medskip

\noindent
{\bf Problem 5.41}

\textit {Solution :}

$\sum_{k}{\binom{n}{k}}{\frac{k!}{(n+1+k)!}}$

expanding the binomial, we have:

$= \sum_{k}{\frac{n!}{(n-k)!k!}}{\frac{k!}{(n+1+k)!}}$

Multiplying and dividing by $(2n+1)!$

$= \sum_{k}{\frac{n!}{(n-k)!k!}}{\frac{k!}{(n+1+k)!}}{\frac{(2n+1)!}{(2n+1)!}}$

$= \sum_{k}{\frac{n!}{(2n+1)!}}{\frac{(2n+1)!}{(n-k)!(n+1+k)!}}$

$= {\frac{n!}{(2n+1)!}}\sum_{k}{\frac{(2n+1)!}{(n-k)!(n+1+k)!}}$

$= {\frac{n!}{(2n+1)!}}\sum_{k=0}^{n}{\binom{2n+1}{n+k+1}}$

Now, if replace (n+k+1) with k

$= {\frac{n!}{(2n+1)!}}\sum_{k=n+1}^{2n+1}{\binom{2n+1}{k}}$

$= {\frac{n!}{(2n+1)!}}\sum_{k=0}^{n}{\binom{2n+1}{k}}$

Adding these two, we get:

$= {\frac{n!}{(2n+1)!}}{(2)^{2n+1}}$

The required result is half the previous result, which gives us:

$= {\frac{n!}{(2n+1)!}}{(\frac{1}{2})(2)^{2n+1}}$

$= {\frac{n!}{(2n+1)!}}{(2)^{2n}}$

\medskip

\medskip

\noindent
{\bf Problem 5.60}

\textit {Solution :}

$\binom{m+n}{n}$

Expanding this, we have:

$= \frac{(m+n)!}{n!m!}$ - This is the form when $m!=n$

$= \frac{(2n)!}{n!n!}$ - This is the form when $m=n$

Using Stirling's approximation:

$\simeq \frac{\sqrt{2\pi (2n)}{\frac{2n}{e}}^{2n} }{(\sqrt{2\pi n} {(\frac{n}{e})}^n{})(\sqrt{2\pi n} {(\frac{n}{e})}^n{})}$

We can simplify this to get :

$\frac{4^n}{\sqrt{\pi n}}$

Which is the required result

Now, when $m!=n$ :

$\simeq \frac{\sqrt{2\pi (m+n)}{\frac{m+n}{e}}^{m+n} }{(\sqrt{2\pi m} {(\frac{m}{e})}^m{})(\sqrt{2\pi n} {(\frac{n}{e})}^n{})}$

$\simeq \frac{{\sqrt{m+n}}{(m+n)}^{m+n}}{{\sqrt{2\pi mn}}{m^m}{n^n}}$

$\simeq {\sqrt{{\frac{m+n}{mn}}{\frac{1}{2\pi}}}}{(\frac{m+n}{m})^m}{(\frac{m+n}{m})^n}$

$\simeq {\sqrt{{\frac{1}{2\pi}}({\frac{1}{m}}+{\frac{1}{n}})}}{(1+\frac{m}{n})^n}{(1+\frac{n}{m})^m}$

Which is the required result

\medskip

\medskip

\noindent
{\bf Problem 5.80}

\textit {Solution :}

To prove:

$\binom{n}{k} \leq (\frac{en}{k})^k$

We can do this via induction:

We can see that this is true when k=1 :

$n\leq en$

Assume this is true for all k. Now, we try and prove it for k+1.

Taking ratios and comparing the rate of growth:

$\frac{\binom{n}{k+1}}{\binom{n}{k}} = \frac{n-k}{k+1}$

$\frac{(\frac{en}{k+1})^{k+1}}{\frac{en}{k}} = (\frac{n}{k+1})(\frac{e}{(1+1/k)^k})$

This is $\geq \frac{n}{k+1}$

Now, clearly, 

$\frac{n}{k+1} \geq \frac{n-k}{k+1}$

Thus, we see that $(\frac{en}{k})^k$ is increasing at a faster rate than $\binom{n}{k}$

Hence, proved...



\end{document}








