\documentclass[12pt]{article}
\usepackage{amsfonts,amsmath,amssymb,graphicx,url}
\usepackage{fullpage}

\def \bangle{ \atopwithdelims \langle \rangle}


% Old Stuff
%%\oddsidemargin=0.15in
%%\evensidemargin=0.15in
%%\topmargin=-.5in
%%\textheight=9in
%%\textwidth=6.25in

\setlength{\oddsidemargin}{.25in}
\setlength{\evensidemargin}{.25in}
\setlength{\textwidth}{6.25in}
\setlength{\topmargin}{-0.4in}
\setlength{\textheight}{8.5in}

\newcommand{\heading}[5]{
   \renewcommand{\thepage}{\arabic{page}}
   \noindent
   \begin{center}
   \framebox{
      \vbox{
    \hbox to 6.2in { {\bf CSE/AMS 547 Discrete Mathematics}
     	 \hfill #2 }
       \vspace{4mm}
       \hbox to 6.2in { {\Large \hfill #5  \hfill} }
       \vspace{2mm}
       \hbox to 6.2in { {\it #3 \hfill #4} }
      }
   }
   \end{center}
   \vspace*{4mm}
}

\newcommand{\handout}[3]{\heading{#1}{#2}{Instructor:
David Gu}{}{#3}}

\setlength{\parindent}{0in}
\setlength{\parskip}{0.1in}

\begin{document}
\handout{2}{December 11, 2017}{Homework 5}

\textbf{Sharad Sridhar - 111492675}

\textbf{Due by Monday, Dec. 11, 11:59pm.}

\noindent
{\bf Problem 6a}

\textit {Solution :}

${n+1 \brace m+1} = \sum_{k=0}^{k=n}{{k \brace m}{(m+1)^{n-k}}}$

Using recurrence on the left side, we have:

${(m+1)}{n \brace m+1}+{n \brace m}$

Applying recurrence again, we have:

${(m+1)}((m+1){n-1 \brace m+1}+{n-1 \brace m})+{n \brace m}$ 

$= {(m+1)^2{n-1 \brace m+1}}+{{(m+1){n-1 \brace m}}+{n \brace m}}$

$= {(m+1)^3{n-2 \brace m+1}}+{{(m+1)^2{n-2 \brace m}}}+{{(m+1){n-1 \brace m}}+{n \brace m}}$

Applying recurrence consecutively, we have:

$(m+1)^n{0 \brace m+1} + \sum_{k=0}^{k=n}{{k \brace m}{(m+1)^{n-k}}}$

The first part always equates to 0 when m > 0 which leads us to the fact that LHS = RHS.

Hence proved.

\medskip

\medskip

\noindent
{\bf Problem 6b}

\textit {Solution :}

${m+n+1 \brace m} = {\sum_{k=1}^{k=m} k{n+k \brace k}}$

We can prove this by induction on m

for m=1, we have LHS:

${n+2 \brace 1} = 1-- if n > 0$

and RHS:

${\sum_{k=0}^{k=m} k{n+k \brace k}} = (1){n+1 \brace 1} = 1-- if n>0$

Assume that it is true for m-1...Now for m, it becomes:

${m-1+n+1 \brace m-1} = {m+n \brace m-1}$

${m+n+1 \brace m} = m{m+n \brace m} +{m+n \brace m-1}$

$= m{m+n \brace m} +  \sum_{k=0}^{k=m-1}{k{n+k \brace k}}$

The RHS will turn out to be : $\sum_{k=0}^{k=m}{k{n+k \brace k}}$

Thus, we have LHS = RHS...

which is the required result

\medskip

\medskip

\noindent
{\bf Problem 6c}

\textit {Solution :}

We start with the recurrence relations for both the sides:

${n \brace k} = {k{n-1 \brace k}} + {n-1 \brace k-1}$

${n \brack k} = {(n-1){n-1 \brack k}} + {n-1 \brack k-1}$

We see that:

${-k \brace -n} = {(-n){-k-1 \brace -n}} + {k-1 \brace -n-1}$

${n \brace k} = {(-n){n \brace k+1}} + {n+1 \brace k+1}$

We now substitute ${-k-1 \brace -n}$ with $n \brack k+1$ and ${-k-1 \brace -n-1}$ with $n+1 \brack k+1$

Now, from the recurrence for the 1st kind, we have:

${n+1 \brace k+1} = {(n+1) - 1 \brack (k+1)-1}+{(n+1-1){(n+1)-1 \brack k+1}}$

$= {n+1 \brack k+1}+{n{n \brack k+1}}$

Hence proved...

\medskip

\medskip

\noindent
{\bf Problem 6.12}

\textit {Solution :}

Let's say that:

if we assume that :

$g(n) = \sum_k{{n \brack k}(-1)^kf(k)}$

we can see that : 

$\sum_k{{n \brack k}(-1)^kg(k)} = \sum_k{n \brack k}(-1)^k{\sum_m{{n \brack m}}(-1)^mf(m)}$

$= \sum_{k,m}{(-1)^{k+m}f(m){n \brack k}{k \brace m}}$

$= \sum_{k,m}{(-1)^{2n-k-m}f(m){n \brack k}{k \brace m}}$

$= \sum_{m}{(-1)^{n-m}f(m)\sum_{k}{(-1)^{n-k}}{n \brack k}{k \brace m}}$

$ = \sum_{m} {(-1)^{n-m}f(m)[m=n]}$

$= f(n)$

Now, if we perform the exact procedure with the parts of f and g reversed, we prove the equation as required.

That is, we assume: $f(n) = \sum_k{{n \brack k}(-1)^kg(k)}$

$\sum_k{{n \brack k}(-1)^kf(k)} = \sum_k{n \brack k}(-1)^k{\sum_m{{n \brack m}}(-1)^mg(m)}$

$= \sum_{k,m}{(-1)^{k+m}g(m){n \brack k}{k \brace m}}$

$= \sum_{k,m}{(-1)^{2n-k-m}g(m){n \brack k}{k \brace m}}$

$= \sum_{m}{(-1)^{n-m}g(m)\sum_{k}{(-1)^{n-k}}{n \brack k}{k \brace m}}$

$ = \sum_{m} {(-1)^{n-m}g(m)[m=n]}$

$= g(n)$

This is the required result.

\medskip

\medskip

\noindent
{\bf Problem 6.15}

\textit {Solution :}

we have : $x^k = \sum_{k}^{}{n \bangle k}{x+k \choose n}$

We know that : $\Delta {x+k \choose n} = {x+k \choose n-1}$

From that we have: ${\delta}^m(x)^n = \sum_{k}{n \bangle k}{x+k \choose n-m}...$

when x=0,

RHS = $\sum_{k} {n \bangle k}{k \choose n-m}...(1)$

Now,

$x^n = \sum_{k} {n \brace k}{x^{\underline{k}}}$

The previous LHS now becomes: $\sum_{k}{n \brace k}(k)(k-1)(k-2)...(k-m+1){x}^{\underline{k-m}}$

When x=0, all terms in the summation are 0 except when $k-m=0$, Thus $k=m$

Now the RHS = ${n \brace m}m(m-1)...1$

$={n \brace m}m!...(2)$

Thus from the previous result , we combine to get:

$m!{n \brace m} = \sum_k{n \bangle k}{k \bangle n-m}$

Which is the required result.

\medskip

\medskip

\noindent
{\bf Problem 6.26}

\textit {Solution :}

We know that: $H_k = H_{k-1} + 1/k$

Now, $H_0 = 0$

From these two, we have :

$S_n = \sum_{k=1}^{k=n}H_{k-1}/k + \sum_{k=1}^{k=n}{1/k^2} = T_n + {H_n}^{(2)}$

Now if we can compute $T_n$, we can compute $S_n$.

For use of summation by parts, we have: $u(k) = H_{k-1}$ and $v(k)=1/k$, then, $\Delta u(k) = 1/k$

We can also set $v(k) = h_{k-1}$, so that: $Ev(k) = H_k$

$\sum_{1}^{n+1} u(x) \Delta v(x) \delta x = u(x)v(x){\Biggr|_{x=n+1}^{x=1}} - \sum_{1}^{n+1}Ev(x) \Delta u(x) \delta x$

$= (H_{x-1})^2 {\Biggr|_{x=n+1}^{x=1}} - \sum_{k=1}^{k=n}{H_k/k}$

$= {H_n}^2-{S_n}$

Now, $S_n = T_n+{H_n}^(2) = {H_n}^2 - S_n + {H_n}^(2)$

From this, we have:

$S_n = ({H_n}^2+{H_n}^{(2)})/2$

\medskip

\medskip

\noindent
{\bf Problem 6.34}

\textit {Solution :}

Using the recurrence relation:

${n \bangle k} = {(k+1){n-1 \bangle k}}+{(n-k){n-1 \bangle k-1}}$

if $k=0$, then ${-1 \bangle k} = 1$

For $k>0$,

${-1 \bangle k} = {\frac{k}{k+1}}{-1 \bangle k-1}$

$= {\frac{k}{k+1}}{\frac{k-1}{k}}{-1 \bangle k-2}$

$= {\frac{k}{k+1}}{\frac{k-1}{k}}...{\frac{1}{2}}{-1 \bangle 0}$

All the terms cancel out to give:

$=\frac{1}{k+1}$

Now, in the original recurrence, we substitute -1 for n, to get:

${-1 \bangle k} = {(k+1){-2 \bangle k}}+{(-1-k){-2 \bangle k-1}}$

Using our previous answer ,we have:

$\frac{1}{k+1} = (k+1){({-2 \bangle k}-{-2 \bangle k-1})}$

$\frac{1}{(k+1)^2} = {({-2 \bangle k}-{-2 \bangle k-1})}$

similar to the previous way, we have:

${-2 \bangle k} = \frac{1}{(k+1)^2}+{-2 \bangle k-1}$

${-2 \bangle k} = \frac{1}{(k+1)^2}+\frac{1}{k^2}+{-2 \bangle k-2}$

${-2 \bangle k} = \frac{1}{(k+1)^2}+\frac{1}{k^2}+...+{-2 \bangle 0}$

${-2 \bangle k} = {H_{k+1}}^{(2)} $

Which is the required result...

\medskip

\medskip

\noindent
{\bf Problem 6.39}

\textit {Solution :}

$\sum_{k=1}^{k=n} {H^2}_k$

$= \sum_{k=1}^{k=n} H_k H_k$

Now, $H_k = \sum_{j=1}^{j=k}{\frac{1}{j}}$

Thus, $= \sum_{k=1}^{k=n} {(\sum_{j=1}^{j=k}{\frac{1}{j}})} H_k$

Switching the sum variables, we have

$= \sum_{j=1}^{k=n} {(1/j)(\sum_{k=j}^{k=n}H_k)}$

$= \sum_{j=1}^{k=n} {(1/j)((\sum_{k=1}^{k=n}H_k)-(\sum_{k=1}^{k=j-1}H_k))}$

Expanding it using the given formula in 6.67, we have:

$= n{H^2}_n - nH_n - \sum {((j-1)H_{j-1}-(j-1))}/j$

$= n{H^2}_n - nH_n - \sum(H_{j-1} - H_{j-1}/j -1 +1/j)$

Now, $H_{n-1} = H-n - 1/n$

$= n{H^2}_n - nH_n - \sum(H_{j} -1/j - (H_{j}-1/j)/j -1 +1/j)$

$= n{H^2}_n - nH_n - \sum(H_{j} -H_j/j + 1/j^2 -1)$

Simplifying  and solving we get the final answer as:

$(n+1){H^(2)}_n - (2n+1)H_n + 2n$

Which is the required result

\noindent
{\bf Problem 6.21} Unable to solve after trying

\end{document}